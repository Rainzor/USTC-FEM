\documentclass{ctexart}
\usepackage{amsmath}
\usepackage{amssymb}
\usepackage[]{hyperref}
\usepackage[a4paper, total={6in, 8in}]{geometry}

\title{FEM - 第三周作业2 Neumann 边值问题}
\author{SA23001071 杨哲睿}
\ctexset{
  section/format    += \sffamily\raggedright,
}

\begin{document}
\maketitle

\section{Introduction}
编写程序求解两点边值问题:
\begin{equation}
  \begin{cases}
    - u'' = f & 0 < x < 1\\
    u'(0) = u'(1) = 0
  \end{cases}
\end{equation}
选取等距网格剖分,有限元空间选取分段线性多项式空间 $V_h$,选取准确解 $u(x) = x^2(x-1)^2$ 算出满足方程的$f(x)$。

\section{Method}

定义线性空间:$V \times \mathbf{R}= H^1([0,1]) \times \mathbb{R}$,定义双线性型:
$$ a((u, c), (v, d)) = \int_0^1 u'(x) v'(x) \mathrm dx + c \int_0^1 v \mathrm dx + d\int_0^1 u \mathrm dx $$
定义内积:
$$ (f, v) = \int_0^1 f(x) v(x) \mathrm dx $$

变分问题为:找 $(u, c) \in V$,使得 $\forall (v, d) \in V$
\begin{equation}
  a((u, c), (v, d)) = (f, v)
\end{equation}

有限维空间:$V_h \times \mathbb{R}$ 其中 $V_h={u: u \in C^0, \text{在每个区间上为分段线性多项式}}$。

实验中采用等距网格划分,节点数为 $N+1$,$0 = x_0 < x_1 < \dots < x_n = 1$,节点处的值为$u_i$,$h = u_{i+1} - u_i$。选取基函数为 $\phi_i$,并选取其张成的有限维线性空间进行求解,全局基函数定义如下:
\begin{equation}
  \phi_0 = \begin{cases}
    \frac{x_{1} - x}{x_{j+1}} & 0 \le x \le x_{1}\\
    0 & \text{otherwise}
  \end{cases}
  \quad
  \phi_i = \begin{cases}
      \frac{x - x_{j-1}}{x_j - x_{j-1}} & x_{j-1} \le x < x_j\\
      \frac{x_{j+1} - x}{x_{j+1} - x_{j}} & x_{j} \le x < x_{j+1}\\
      0 & \text{otherwise}
  \end{cases}
  \quad
  \phi_n = \begin{cases}
    \frac{x-x_{n-1}}{1-x_{n-1}} & x_{n-1} \le x \le 1\\
    0 & \text{otherwise}
  \end{cases}
\end{equation}

对于单元$[x_{i-1}, x_i]$局部基函数为:
\begin{equation}
    \phi_0^i (x) =\begin{cases}
        \frac{x - x_{i-1}}{x_{i}- x_{i-1}}, & x_{i-1} \le x \le x_i\\
        0, & \text{otherwise}
    \end{cases}
    \phi_1^i (x) =\begin{cases}
        \frac{x_{i} - x}{x_{i}- x_{i-1}}, & x_{i-1} \le x \le x_i\\
        0, & \text{otherwise}
    \end{cases}
\end{equation}

标准单元$[0, 1]$对应的形函数为:
\begin{equation}
    \phi_0 (x) = \begin{cases}
        x, & 0 < x < 1\\
        0, & \text{otherwise}
    \end{cases}
    \phi_1 (x) = \begin{cases}
        1-x, & 0 < x < 1\\
        0, & \text{otherwise}
    \end{cases}
\end{equation}

局部刚度矩阵的系数计算:
\begin{equation}
    \begin{aligned}
        K_{0, 0} &= a(\phi_0^i, \phi_0^i) = \int_{x_{i-1}}^{x_i} \frac{1}{h^2} \mathrm dx = \frac{1}{h}\\
        K_{1, 1} &= a(\phi_1^i, \phi_1^i) = \int_{x_{i-1}}^{x_i} \frac{1}{h^2} \mathrm dx = \frac{1}{h}\\
        K_{0, 1} &= K_{1, 0} = a(\phi_0^i, \phi_1^i) = \int_{x_{i-1}}^{x_i} -\frac{1}{h^2} \mathrm dx = -\frac{1}{h}
    \end{aligned}
\end{equation}
因此局部刚度矩阵为:
\begin{equation}
    K = \begin{pmatrix}
        K_{0, 0} & K_{0, 1}\\
        K_{1, 0} & K_{1, 1}
    \end{pmatrix} = \frac{1}{h} \begin{pmatrix}
        1 & -1\\ -1 & 1
    \end{pmatrix}
\end{equation}

全局刚度矩阵:
\begin{equation}
    A = \sum_{i = 1} ^ {N+1} K_i 
\end{equation}
从而:
\begin{equation}
    A = \frac{1}{h}\begin{pmatrix}
          1 & -1 &  \cdots &0 & 0 \\
         -1 & 2  &  \cdots &0 & 0\\
          0 & \vdots &  & \vdots &0 \\
          0 & 0 & \cdots& 2 & -1\\
          0 & 0 & \cdots& -1 & 1\\
    \end{pmatrix}
\end{equation}

计算 $(f, \phi_j)$ 得:
\begin{equation}
\begin{aligned}
    f_j &= \int_0 ^ 1 f \phi_j \mathrm dx = h(f(x_j) + O(h))
\end{aligned}
\end{equation}

其次,计算如下的:
\begin{equation}
  d \int_0^1 v \mathrm dx = \sum_{i=0}^{N}dv(x_i)\int_{0}^1 \phi_i(x)\mathrm dx
\end{equation}

令 
\begin{equation}
  \mathbf{V} = \begin{pmatrix}
    v(x_0)\\
    v(x_1)\\
    \vdots\\
    v(x_N)\\
    d
  \end{pmatrix}
  \mathbf{U} = \begin{pmatrix}
    u(x_0)\\
    u(x_1)\\
    \vdots\\
    u(x_N)\\
    c
  \end{pmatrix}
  \mathbf{F} = \begin{pmatrix}
    f_0\\
    f_1\\
    \vdots\\
    f_N\\
    0
  \end{pmatrix}
\end{equation}
那么 $a((u, c), (v, d)) = (f, v)$可以写为:
\begin{equation}
  \mathbf{V}^T \mathbf{KU}=\mathbf{V}^T \begin{pmatrix}
    A_{11} & \dots & A_{1N} & \int_0^1 \phi_1 \mathrm dx\\
    \vdots & \ddots & \vdots & \vdots \\
    A_{N1}& \dots & A_{NN} & \int_0^1 \phi_N \mathrm dx\\
    \int \phi_1 \mathrm dx & \dots & \int_0^1 \phi_N \mathrm dx& 0
  \end{pmatrix}
  \mathbf{U} = \mathbf{V}^T\mathbf{F}
\end{equation}

因此,求解如下的线性方程组:
\begin{equation}
  \mathbf{KU}=\mathbf{F}
\end{equation}

\section{Results}

实验选取 $u(x) = x^2(x-1)^2$,那么 $-u''(x) = -2 x (1 - 3 x + 2 x^2)$。对于求得的系数,先进行插值,然后进行梯形公式数值积分。求得的 $\int_0^1 u(x)\mathrm dx=0$,因此精确解应为 $x^2(x-1)^2-\frac{1}{30}$。
\begin{table}[h!]
  \caption{\label{table.label} 误差分析} \centering
  \bigskip
  \begin{small}
  \begin{tabular}{|c|cc|cc|}
    \hline
    % after \\: \hline or \cline{col1-col2} \cline{col3-col4} ...
    n &$L^2 $ error & order &$ H^1$ error & order\\\hline
    10 & 3.402e-03 & --- & 1.021e-01 & --- \\
    20 & 8.557e-04 & 1.99 & 5.092e-02 & 1.004\\
    40 & 2.142e-04 & 1.99 & 2.518e-02 & 1.016 \\
    80 & 5.359e-05 & 1.99 & 1.229e-02 & 1.034\\\hline
  \end{tabular}
  \end{small}
\end{table}
可以看出,该方法对$L^2$误差为2阶,对$H^1$误差为一阶。其理由与前两次实验相同。


\section*{Reference}

\href[]{https://fenicsproject.org/olddocs/dolfin/1.6.0/python/demo/documented/neumann-poisson/python/documentation.html}{FEniCS -- 11. Poisson equation with pure Neumann boundary conditions}

\end{document}
