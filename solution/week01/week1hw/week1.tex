\documentclass[11pt]{ctexart}
\title{FEM 书面作业 1}
\usepackage{amssymb}
\usepackage{amsmath}

\ctexset {
     section/format    += \sffamily\raggedright,
     subsection/format += \fbox,
}


\author{SA23001071 杨哲睿}
\date{\today}
\begin{document}

\maketitle

\section{习题 1}

假设 $V = \{ w | w \in C [0, 1] $, $w'$ 是 $[0, 1]$ 中的分片连续有界函数, $w(0) = w(1) = 0\}$,$w \in C[0, 1]$ 满足 
$$ \int _0 ^ 1w v \mathrm d x = 0,\quad \forall v \in V $$
证明:
$$ w(x) = 0, \quad \forall x \in [0, 1]$$

{\flushleft \textbf{证明}:}

假设 $\exists x_0\in (0, 1), w(x_0) > \alpha > 0$ 因为 $w$ 是连续函数,所以 $\exists B(x_0, r) \subset (0,1)$ 使得 $\forall x \in B(x_0, r)$ 有 $f(x) > \alpha$,这是因为连续函数值域中开集的原像是开集。

考虑如下的函数:

\begin{equation}
  v = \begin{cases}
    \frac{1}{r}(x- x_0 + r)^2, & x_0 - r \le x < x_0 - \frac{1}{2} r\\
    - \frac{1}{r}(x - x_0) ^ 2 + \frac{r}{2}, & x_0 - \frac 1 2 r \le x < x_0 + \frac 1 2 r \\
    \frac{1}{r}(x- x_0 - r)^2, & x_0 - r \le x < x_0 - \frac{1}{2} r\\
    0, & \text{otherwise}
  \end{cases}
\end{equation}

不难验证:
\begin{equation}
  v' = \begin{cases}
    \frac{2}{r}(x-x_0+r),  & x_0 - r \le x < x_0 - \frac{1}{2} r\\
    - \frac{2}{r}(x - x_0), & x_0 - \frac 1 2 r \le x < x_0 + \frac 1 2 r \\
    \frac{2}{r}(x- x_0 - r), & x_0 - r \le x < x_0 - \frac{1}{2} r\\
    0, & \text{otherwise}
  \end{cases}
\end{equation}
是分片连续的函数,并且 $v(0) = v(1) = 1$,因此 $v \in V$。
\begin{equation}
  0 = \int_0 ^ 1 w v \mathrm dx > \alpha \int_{x_0 - r} ^ {x_0+r} v \mathrm dx > 0 
\end{equation}
产生矛盾。

因此 $\forall x \in (0, 1), w(x) = 0$,由$w$的连续性可知,$w(0) = w(1) = 0$。综上:

$$ w(x) = 0\quad \forall x \in [0, 1] $$

\section{习题2}

设$f(x)$是光滑函数,给出两点边值问题
\begin{equation}
  \begin{cases}
    -u'' + u = f,& 0 < x < 1\\
    u(0) = u(1) = 0
  \end{cases}
\end{equation}
对应的变分问题。

{\flushleft \textbf{解}:}

定义双线性型 $a(u, v) := \int_0 ^ 1 u' v' \mathrm dx$.

定义空间 $V = \left\{ v: v \in L^2(0, 1), a(v, v) < \infty, v(0) = v(1) = 0\right\}$.

对应的变分问题为:求 $u \in V$,使得
\begin{equation}
  a(u,v) = (f, v)\quad \forall v \in V
\end{equation}

\section{习题3}


设$f(x)$是光滑函数,给出两点边值问题
\begin{equation}
  \begin{cases}
    -(a(x) u'(x)) + u (x)= f,& 0 < x < 1\\
    u(0) = u'(1) = 0
  \end{cases}
\end{equation}
对应的变分问题。

{\flushleft \textbf{解}:}

定义双线性型 $a(u, v) := \int_0 ^ 1 a u' v'+ uv \mathrm dx$.

定义空间 $V = \left\{ v: v \in L^2(0, 1), a(v, v) < \infty,v(0) = 0\right\}$.

此时:
\begin{equation}
  \begin{aligned}
    & \int_0^1 f v \mathrm dx\\
   =&\int_0^1 - v(au')' + uv \mathrm dx\\
   =&-(v(x)a(x)u'(x))|^1_0 + \int_0^1 au'v'\mathrm dx + \int_0^1 uv \mathrm dx\\
   =&\int_0^1 au'v'\mathrm dx + \int_0^1 uv \mathrm dx\\
   =& a(u,v)
  \end{aligned}
\end{equation}
从而,对应的变分问题为:求 $u \in V$,使得
\begin{equation}
  a(u,v) = (f, v)\quad \forall v \in V
\end{equation}



\section{习题4}

假设函数 $f$ 是分片线性的,$f(x) = \sum_{j = 1}^N f_j \phi_j(x)$,证明:求解边值问题的有限元方法可以写成如下形式:
$$ A U = M F $$
其中$M$是质量矩阵.

{\flushleft \textbf{证明}:}假设求解的边值问题本身可以转化为一个变分问题,即存在双线性型$a(u,v)$、内积 $( \cdot, \cdot)$ 和空间$V$使方程求解问题的解等价于找 $u\in V$,使得$a(u,v) = (f, v) \forall v \in V$。

取$V_h$是$V$的有限维子空间,$\{\psi_i\}_{i=1}^M$是$V_h$上的一组基。在子空间 $V_h$中,$u = \sum_{i=1}^M u_i \psi_i, v = \sum_{i = 1}^M v_i \psi_i$,那么双线性型:
\begin{equation}
\begin{aligned}
  a(u, v) &= a(\sum_{i=1}^N u_i \psi_i(x), \sum_{j=1}^N v_j \psi_j(x))
  &= \sum_{i=1}^N \sum_{j=1}^N u_i v_j a(\psi_i, \psi_j)
\end{aligned}
\end{equation}
定义矩阵 $A\in \mathbb{R}^{M\times M}$,其中每个元素 $a_{ij} = a(\psi_i, \psi_j)$,向量$W = (v_1, v_2, \cdots, v_M)^T$,$U = (u_1, u_2, \cdots, u_M)^T$,那么
\begin{equation}
  a(u,v) = W^T A U
\end{equation}

同时\begin{equation}
\begin{aligned}
  (f, v) &= \left(\sum_{j=1}^N f_j \phi_j(x), \sum_{i=1}^M v_i \psi_i(x)\right)\\
  &= \sum_{j=1}^N\sum_{i=1}^M (\phi_j(x), \psi_i(x))
\end{aligned}
\end{equation}

令 $F = (f_1, f_2, \cdots, f_N)^T$,矩阵 $M\in \mathbb{R}^{M \times N}$,其中每个元素为 $m_{ij} = (\phi_j(x), \psi_i(x))$,那么:
\begin{equation}
  (f,v) = W^T M F
\end{equation}

此时,变分问题转化为 $W^T A U = W^T M F,\forall W$,那么
\begin{equation}
  AU = MF\label{result}
\end{equation}
这说明了求解边值问题的有限元方法能够写成\eqref{result}的形式.


\end{document}




















