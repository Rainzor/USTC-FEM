\documentclass[11pt]{ctexart}

\usepackage{amsmath}
\usepackage{amssymb}

\title{FEM 第二次 书面作业}
\author{杨哲睿 SA23001071}

\begin{document}
\maketitle
\section{习题1}
证明如下命题:假设$f$充分光滑,那么
\begin{equation}
  (f, \phi_i) = \frac{1}{2} (h_i + h_{i+1}) (f(x_i) + O(h))
\end{equation}
其中 $h = \max h_i$

{\flushleft \textbf{证明}:} 因为$f$充分光滑,所以存在在 $x_i$ 处的Taylor展开式如下:
\begin{equation}
  f(x_i + s) = f(x_i) + f'(x_i) s + O(s^2) = f(x_i) + O(s)
\end{equation}
因此:
\begin{equation}
  \begin{aligned}
    (f, \phi_i) & = \int_0^1 f(x) \phi_i(s) \mathrm ds\\
     & = \int_{-h_i}^{h_{i+1}} f(x_i + s) \phi(x_i + s) \mathrm ds \\
     & = \int_{-h_i}^{h_{i+1}} f(x_i) \phi(x_i + s) + \phi_i(x_i + s) O(h) \mathrm ds
  \end{aligned}
\end{equation}
因为 $h = \max_i h_i$,所以积分式中 $| s | < h$:
\begin{equation}
  \begin{aligned}
    (f, \phi_i) & = \frac{1}{2}(h_i+h_{i+1}) (f(x_i) + 2 O(h))\\
    & = \frac{1}{2}(h_i+h_{i+1}) (f(x_i) + O(h))
  \end{aligned}
\end{equation}

\begin{flushright}
$\square$
\end{flushright}

\section{习题2}

设 $h = \max_{1 \le i \le n} (x_i - x_{i-1})$,证明
\begin{equation}
  \| u - u_I \| \le C h^2 \| u''\|
\end{equation}

提示:利用$w(0) = 0$说明:
\begin{equation}
  \int_0^1 w(x)^2 \mathrm dx \le \bar{c} \int_0^1 w'(x)^2 \mathrm dx 
\end{equation}
以及在$w(0) = w(1) = 0$的前提下,寻找尽量小的$\bar{c}$。

\begin{flushleft}
\textbf{证明}:
\end{flushleft}

首先考虑提示的内容,先考虑$w(0) =0$的情况:
\begin{equation}
  \begin{aligned}
    w(t) &=\int_0^t w'(x)\mathrm dx \\
    &= \int_0^t 1 \cdot w'(x)\mathrm dx \\
    \text{(By Cauchy Inequality)}&\le \left( \int_0^t 1^2 \mathrm dx \right)^{1/2}
    \left( \int_0^t w'(x)^2 \mathrm d x\right)^{1/2}\\
  \end{aligned}
\end{equation}
因此
\begin{equation}
  w(t)^2 \le t \int_0^t w'(x)^2\mathrm dx
\end{equation}
因此:
\begin{equation}
  \begin{aligned}
    \int_0^1 w(x)^2 \mathrm dx &\le \int_0^1 \left(t \int_0^t w'(x)^2\mathrm dx\right) \mathrm dt\\
    &\le \| t \|_1 \| \int_0^t w'(x)^2\mathrm dx\|_\infty\\
    &= \frac{1}{2} \int_0^1 w'(x)^2\mathrm dx
  \end{aligned}
\end{equation}
那么,在$w(0) = 0$的情况下,该不等式成立,且$\bar c = \frac{1}{2}$

$w(0) = w(1) = 0$ 的情况,类似的有:
\begin{equation}
  \begin{aligned}
    w(t) &=  \int_1^t w'(x)\mathrm dx \\
    &\le \left( \int_t^1 1^2 \mathrm dx \right)^{1/2}
    \left( \int_t^1 w'(x)^2 \mathrm d x\right)^{1/2}\\
  \end{aligned}
\end{equation}
因此
\begin{equation}
  w(t)^2 \le (1-t) \int_t^1 w'(x)^2\mathrm dx
\end{equation}
那么:
\begin{equation}
  \int_t^1 w^2 \mathrm dx\le \frac{1}{2}(1-t)^2 \int_t^1 w'(x)^2 \mathrm dx
\end{equation}

那么
\begin{equation}
  \begin{aligned}
    2\int_0^1 w(x)^2 \mathrm dx& = 
            \int_0^{\alpha} w(x)^2 \mathrm dx + 
            \int_{\alpha}^1 w(x)^2 \mathrm dx\\
    &\le 
        \alpha^2\int_0^{\alpha} w'(x)^2\mathrm dx 
        + (1-\alpha)^2\int_{\alpha}^{1} w'(x)^2\mathrm dx
  \end{aligned}
\end{equation}


令$F = \int_0^{\alpha} w'(x)^2\mathrm dx$,$M = \int_0^{1} w'(x)^2\mathrm dx$,那么:
\begin{equation}
  2\int_0^1 w(x)^2 \mathrm dx \le \alpha^2 F + (1 - \alpha)^2 (M-F)
\end{equation}
右侧函数取最大值时,$\alpha = \frac{1}{2}$,$F = M /2$,这说明:
\begin{equation}
  2\int_0^1 w(x)^2 \mathrm dx \le M/4 = \frac{1}{4}\int_0^1 w'(x)^2\mathrm dx
\end{equation}
这说明在$w(0) = w(1) = 0$的情况下,$\bar c = \frac{1}{8}$

下面证明原本的定理。令$e(x) = u(x) - u_I(x)$,考察 $[x_{i-1}, x_{i+1}]$,满足:
\begin{equation}
  e(x_{i-1}) = e(x_i) = e(x_{i+1})
\end{equation}
这说明,存在$\xi_{i-1}, \xi_{i}$使得$e'(\xi_{i-1})=e'(\xi_{i}) = 0$

利用证明的引理:
\begin{equation}
  \begin{aligned}
    \int_{\xi_{i-1}}^{\xi_i} e'(x)^2 \mathrm dx &\le \frac{1}{4} (\xi_{i}-\xi_{i-1})\int_{\xi_{i-1}}^{\xi_i} e''(x)^2 \mathrm dx\\
    &\le h\int_{\xi_{i-1}}^{\xi_i} e''(x)^2 \mathrm dx
  \end{aligned}
\end{equation}
那么,对于$e$有:
\begin{equation}
  \begin{aligned}
    \int_{\xi_{i-1}}^{\xi_i} e(x)^2 \mathrm dx &\le \frac{1}{8} h^2\int_{\xi_{i-1}}^{\xi_i} e'(x)^2 \mathrm dx
  \end{aligned}
\end{equation}
与此同时,可以对$e, e'$也建立不等式:
\begin{equation}
  \int_{x_{i-1}}^{x_i} e(x)^2 \mathrm dx \le \frac{1}{8} h^2\int_{x_{i-1}}^{x_i} e'(x)^2 \mathrm dx
\end{equation}

对 $i$ 求和:
\begin{equation}
  \begin{aligned}
    \int_0^1 e(x)^2 \mathrm dx &\le (\frac{1}{8}h^2) 
    \sum_{i=1}^N\int_{x_{i-1}}^{x_i} e'(x)^2\mathrm dx\\
    &\le(\frac{1}{8}h^2) \sum_{i=1}^N(\frac{1}{8}h^2) \int_{\xi_{i-1}}^{\xi_i} e''(x)^2\mathrm dx
  \end{aligned}
\end{equation}
考虑到 $u_I'' = 0$ 立刻得到:
\begin{equation}
  \| u - u_I \| \le\frac{1}{8}  h^2 \| u'' \|
\end{equation}

\begin{flushright}
  $\square$
\end{flushright}

\end{document}
