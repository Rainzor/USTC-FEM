\documentclass{ctexart}
\usepackage{amsmath}
\usepackage{amssymb}
\usepackage{subfigure}

\usepackage{verbatim}
\usepackage{fancyhdr}


\usepackage{graphicx} % Required for inserting images
\ctexset {
     section/format    += \sffamily\raggedright,
}

\title{有限元方法 代码作业2 报告}
\author{SA23001071 杨哲睿 }
\date{\today}

\begin{document}

\maketitle

\section{Introduction}

编写程序求解两点边值问题
\begin{equation}
\begin{cases} 
    -u'' = f, &\quad 0 < x < 1\\
    u(0) = u(1) = 0
\end{cases}\label{orig-problem}
\end{equation}
选取等距网格剖分,有限元空间选取分段线性多项式空间$V_h$。其中,选取了$f(x) = -(2 \cos x - (x - 1) \sin x)$,已知其解为$u(x) = (x-1) \sin x$。完成求解后,使用$L^2$范数和$H^1$范数计算误差并讨论结果。

\section{Method}

定义双线性型$a(u, v) = \int_0^1 u' v' \mathrm dx$,内积$(f, g) = \int_0 ^ 1 f \cdot g \mathrm dx$。那么问题 \eqref{orig-problem} 的变分问题是

\textit{找 $u\in V = \{ v\in C[0, 1], v(0) = v(1) = 0\}$,使得$a(u, v) = (f, v), ~ \forall v \in V$都成立.}

实验中采用等距网格划分,节点数为 $N$,节点处的值为$u_i$,$h = u_{i+1} - u_i$。选取基函数为 $\phi_i$,并选取其张成的有限维线性空间进行求解,全局基函数定义如下:
\begin{equation}
    \phi_i = \begin{cases}
        \frac{x - x_{j-1}}{x_j - x_{j-1}} & x_{j-1} \le x < x_j\\
        \frac{x_{j+1} - x}{x_{j+1} - x_{j}} & x_{j} \le x < x_{j+1}\\
        0 & \text{otherwise}
    \end{cases}
\end{equation}

对于单元$[x_{i-1}, x_i]$局部基函数为:
\begin{equation}
    \phi_0^i (x) =\begin{cases}
        \frac{x - x_{i-1}}{x_{i}- x_{i-1}}, & x_{i-1} \le x \le x_i\\
        0, & \text{otherwise}
    \end{cases}
    \phi_1^i (x) =\begin{cases}
        \frac{x_{i} - x}{x_{i}- x_{i-1}}, & x_{i-1} \le x \le x_i\\
        0, & \text{otherwise}
    \end{cases}
\end{equation}

标准单元$[0, 1]$对应的形函数为:
\begin{equation}
    \phi_0 (x) = \begin{cases}
        x, & 0 < x < 1\\
        0, & \text{otherwise}
    \end{cases}
    \phi_1 (x) = \begin{cases}
        1-x, & 0 < x < 1\\
        0, & \text{otherwise}
    \end{cases}
\end{equation}

局部刚度矩阵的系数计算:
\begin{equation}
    \begin{aligned}
        K_{0, 0} &= a(\phi_0^i, \phi_0^i) = \int_{x_{i-1}}^{x_i} \frac{1}{h^2} \mathrm dx = \frac{1}{h}\\
        K_{1, 1} &= a(\phi_1^i, \phi_1^i) = \int_{x_{i-1}}^{x_i} \frac{1}{h^2} \mathrm dx = \frac{1}{h}\\
        K_{0, 1} &= K_{1, 0} = a(\phi_0^i, \phi_1^i) = \int_{x_{i-1}}^{x_i} -\frac{1}{h^2} \mathrm dx = -\frac{1}{h}
    \end{aligned}
\end{equation}
因此局部刚度矩阵为:
\begin{equation}
    K = \begin{pmatrix}
        K_{0, 0} & K_{0, 1}\\
        K_{1, 0} & K_{1, 1}
    \end{pmatrix} = \frac{1}{h} \begin{pmatrix}
        1 & -1\\ -1 & 1
    \end{pmatrix}
\end{equation}

对于内部的单元 $[x_{i-1}, x_{i}]$,其对应到全局的矩阵为:
\begin{equation}
    K_i = \bordermatrix{
    ~   & &  & i-1 & i \cr
    ~   & 0 & \cdots & 0 & 0 & \cdots & 0\cr
    ~   & \vdots & \ddots & \vdots & \vdots & \ddots & \vdots\cr
    ~   & 0 & \cdots & 0 & 0 & \cdots & 0\cr
    i-1 & 0 & \cdots & \frac{1}{h} & -\frac{1}{h}& \cdots & 0 \cr
    i & 0 & \cdots & -\frac{1}{h} & \frac{1}{h}& \cdots & 0 \cr
    ~   & 0 & \cdots & 0 & 0 & \cdots & 0\cr
    ~   & \vdots & \ddots & \vdots & \vdots & \ddots & \vdots\cr
    ~   & 0 & \cdots & 0 & 0 & \cdots & 0\cr}_{N \times N}
\end{equation}

对于边界上的 $[0, x_1]$:
\begin{equation}
    K_1 = \begin{pmatrix}
        \frac{1}{h} & 0 & \cdots & 0\\
        0 & 0 & \cdots & 0 \\
        \vdots & \vdots & \ddots & \vdots \\
        0 & 0 & \cdots & 0
    \end{pmatrix}_{N \times N}
\end{equation}
对于$[x_N, 1]$:
\begin{equation}
    K_{N+1} = \begin{pmatrix}
        0 & \cdots & 0& 0\\
        \vdots & \ddots & \vdots & \vdots \\
        0 & \cdots & 0& 0 \\
        0 & \cdots & 0 & \frac{1}{h}
    \end{pmatrix}_{N \times N}
\end{equation}

全局刚度矩阵:
\begin{equation}
    A = \sum_{i = 1} ^ {N+1} K_i 
\end{equation}
从而:
\begin{equation}
    A = \frac{1}{h}\begin{pmatrix}
          2 & -1 &  \cdots &0 & 0 \\
         -1 & 2  &  \cdots &0 & 0\\
          0 & \vdots &  & \vdots &0 \\
          0 & 0 & \cdots& 2 & -1\\
          0 & 0 & \cdots& -1 & 2\\
    \end{pmatrix}
\end{equation}

计算 $(f, \phi_j)$ 得:
\begin{equation}
\begin{aligned}
    f_j &= \int_0 ^ 1 f \phi_j \mathrm dx\\
    &= \frac{4 (h j-1) \sin ^2\left(h/2\right) \sin (h j)}{h}-2 \sin (h) \cos (h j)
\end{aligned}
\end{equation}

因此为求解节点上的值,只需要求解出 $A U = F$,其中:
\begin{equation}
    U  = \begin{pmatrix}
        u_1\\
        \vdots\\
        v_N
    \end{pmatrix},
    \quad
    F = \begin{pmatrix}
        f_1\\
        \vdots\\
        f_N
    \end{pmatrix}
\end{equation}

在估计$L^2$和$H^1$误差进行时,先对于求出的$u_h$进行线性插值,然后再使用梯形公式来进行近似积分值,不存在较大的精度问题。

\section{Results}

实验得到的结果如下:
\newpage

\begin{table}[h!]
\caption{\label{table.label} 误差分析表} \centering
\bigskip
\begin{small}
\begin{tabular}{|c|cc|cc|}
  \hline
  % after \\: \hline or \cline{col1-col2} \cline{col3-col4} ...
  n &$L^2 $ error & order &$ H^1$ error & order\\\hline
10&   1.41E-03  &  --   &  4.90E-02 &  -- \\
20&   3.86E-04  &  1.87 & 2.56E-02 &  0.93\\
40&   1.01E-04  &  1.93 & 1.31E-02 &  0.96 \\
80&   2.6E-05  &  1.95 & 6.62E-3 &  0.98\\\hline
\end{tabular}
\end{small}
\end{table}

观察到,该方法对 $L^2$ 误差具有二阶精度,对$H^1$误差具有一阶精度. 

\section{Discussion}
本节着重进行误差分析,令 $v \in V_h$ 为一分片线性函数,满足:
\begin{equation}
    v(x_i) = u(x_i)\quad \forall i = 1, 2, \cdots, N
\end{equation}
其中 $u$ 是两点边值问题的古典解。令 $e = v - u$,那么

考虑区间 $[a,b]$ 上的$C^1$函数 $f$,满足$f(a) = f(b) = 0$,那么$|f| < \frac{b-a}{2} \max|f'|$,积分
\begin{equation}
    \int_a^b |f(x)| \mathrm d x < \frac{(b-a)^2}{2} \max | f'|
\end{equation}

对 $e, e'$ 反复利用该结论以及微分中值定理可知,对于 $e, e'$ 有如下的最大误差估计:
\begin{equation}
\begin{cases}
    \sup _ x | e' | \le C_1 h \sup_x | u''|\\
    \sup _ x | e | \le C_2 h ^ 2 \sup_x | u''|
\end{cases}
\end{equation}
那么
\begin{equation}
\begin{cases}
    || u - v||_2 = \le C_1 \cdot h^2 \sup_x|u'' |\\
    || u' - v'||_2 = \le C_2 \cdot h \sup_x|u'' |\\
\end{cases}
\end{equation}

代入$L^2$误差和$H$误差的表达式中可知,其在 $L^2$ 意义下有二阶精度,$H^1$意义下有一阶精度。

\begin{flushleft}
    这说明了,第一次作业存在问题。在排查之后,发现在$H^1$范数的计算上出现了明显的错误。
\end{flushleft}


\end{document}
