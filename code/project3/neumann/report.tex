\documentclass[11pt]{ctexart}
% \documentclass{article}
\textheight 23.5cm \textwidth 15.8cm
%\leftskip -1cm
\topmargin -1.5cm \oddsidemargin 0.3cm \evensidemargin -0.3cm

\usepackage{verbatim}
\usepackage{fancyhdr}
\usepackage{graphicx}
\usepackage{amssymb}
\usepackage{amsmath}
\usepackage{booktabs}
\usepackage{subcaption}
\usepackage{listings}
\usepackage{color}
\usepackage{geometry}


\definecolor{mygreen}{rgb}{0,0.6,0}
\definecolor{mygray}{rgb}{0.5,0.5,0.5}
\definecolor{mymauve}{rgb}{0.58,0,0.82}

\lstset{
  language=Matlab,                % 设定语言为MATLAB
  frame=single,                   % 外围框架
  basicstyle=\footnotesize\ttfamily,   % 基本代码风格
  keywordstyle=\color{blue},      % 关键词风格
  commentstyle=\color{mygreen},   % 注释风格
  stringstyle=\color{mymauve},    % 字符串风格
  numbers=left,                   % 行号位置
  numberstyle=\tiny\color{mygray}, % 行号风格
  stepnumber=1,                   % 行号步长
  numbersep=5pt,                  % 行号与代码间隔
  backgroundcolor=\color{white},  % 代码背景色
  showspaces=false,               % 不显示空格
  showstringspaces=false,         % 不显示字符串中的空格
  showtabs=false,                 % 不显示制表符
  tabsize=4,                      % 制表符宽度
  captionpos=b,                   % 标题位置
  breaklines=true,                % 自动换行
  breakatwhitespace=false,        % 仅在空格处换行
  escapeinside={\%*}{*)},         % 可以添加LaTeX内容
}


\ctexset {
     section/format    += \sffamily\raggedright,
     subsection/format += \fbox,
}

\title{FEM Code Report3 - Neumann}
\author{SA24229016 王润泽}

\begin{document}
\maketitle

\section{Introduction}
编写程序求解两点边值问题:
\begin{equation}
     \begin{aligned}
          -u'' &= f, \quad 0 < x < 1, \\
          u'(0) &= u'(1) = 0.
     \end{aligned}
\end{equation}
选取等距网格剖分,有限元空间选取为分段连续线性多项式空间,每个单元的基函数利用单元$ I_j = [x_{j-1},x_j] $端点$x_{j-1}, x_j$得到。

\section{Method}

\subsection{Variational Formulation}

定义线性空间:$ V\times R = H^1_{[0,1]}\times \mathbb{R} $。

对于任意$ (u,c)\in V\times R , (v,d)\in V\times R $,定义双线性形式:
\begin{equation*}
     a((u,c),(v,d)) = \int_0^1 u'v' \,dx + c\int_0^1 v \,dx + d\int_0^1 u \,dx.
\end{equation*}

定义内积:
\begin{equation*}
     (f,v) = \int_0^1 f v \,dx 
\end{equation*}

那么原问题转换成变分问题为:求$ (u,c)\in V\times R $,使得对于所有$ (v,d)\in V\times R $,有:
\begin{equation}
     a((u,c),(v,d)) = (f,v).
\end{equation}

\subsection{FEM基函数}
实验中采用等距网格划分,节点数为N+1, $ 0=x_0<x_1<\dots<x_{N}=1 $, 节点处的值为 $ u_i $,步长为$h_i = x_{i}-x_{i-1}$ 。选取分片线性全局基函数$ \phi_i(x) $,并选取其张成的有限维线性空间进行求解,其定义如下:
\begin{equation}
  \phi_i(x) = \left\{
       \begin{aligned}
            &\frac{x-x_{i-1}}{h_i}, \quad x \in I_{i}, \\
            &\frac{x_i-x}{h_{i+1}}, \quad x \in I_{i+1}, \\
            &0, \quad \text{otherwise}.
       \end{aligned}
  \right.
  \quad i=1,2,\dots,N-1
\end{equation}

\begin{equation}
  \phi_0(x) = \left\{
       \begin{aligned}
            & \frac{x_1-x}{h_1}, \quad x \in I_1, \\
            & 0, \quad \text{otherwise}.
        \end{aligned}
      \right.,\quad 
  \phi_N(x) = \left\{
        \begin{aligned}
             & \frac{x-x_{N-1}}{h_N}, \quad x \in I_N, \\
             & 0, \quad \text{otherwise}.
         \end{aligned}
         \right.
\end{equation}

对于单元 $ I_i = [x_{i-1},x_i] $,其局部基函数 $ \psi $ 定义如下:
\begin{equation}
  \psi^0_i(x) = \left\{
       \begin{aligned}
            &\frac{x_i-x}{h_i}, \quad x \in I_{i}, \\
            &0, \quad \text{otherwise}.
       \end{aligned}
  \right.,\quad
  \psi^1_i(x) = \left\{
       \begin{aligned}
            &\frac{x-x_{i-1}}{h_i}, \quad x \in I_{i}, \\
            &0, \quad \text{otherwise}.
       \end{aligned}
  \right.
\end{equation}

对于标准单元 $ I = [0,1] $,其基函数 $ {\varphi} $ 定义如下:
\begin{equation}
  {\varphi}^0(x) = \left\{
       \begin{aligned}
            &1-x, \quad x \in [0,1], \\
            &0, \quad \text{otherwise}.
       \end{aligned}
  \right.,\quad
  {\varphi}^1(x) = \left\{
       \begin{aligned}
            &x, \quad x \in [0,1], \\
            &0, \quad \text{otherwise}.
       \end{aligned}
  \right.
\end{equation}

\subsection{刚度矩阵}
局部刚度矩阵描述了双线性形式 $ a(u,v) $ 在局部基函数上的表现,其定义如下:
\begin{equation}
  \begin{aligned}
       k_{i}^{00}&= a(\psi_i^0,\psi_i^0) = \int_{I_i} \psi_i^0(x)' \psi_i^0(x)' = \frac{1}{h_i}, \\
       k_{i}^{01}&= k_{i}^{10}= a(\psi_i^0,\psi_i^1) = \int_{I_i} \psi_i^0(x)' \psi_i^1(x)' = -\frac{1}{h_i}, \\
       k_{i}^{11}&= a(\psi_i^1,\psi_i^1) = \int_{I_i} \psi_i^1(x)' \psi_i^1(x)' = \frac{1}{h_i}.
  \end{aligned}
\end{equation}

对于等分剖分的单元,其刚度矩阵 $ K_i$ 定义如下:
\begin{equation}
  K_i = \frac{1}{h} \begin{pmatrix}
       1 & -1 \\
       -1 & 1
  \end{pmatrix}.
\end{equation}

得到全局刚度矩阵 $ A $ 为:
\begin{equation}
  A = \sum_{i=1}^{N+1} K_i\
\end{equation}
即:
\begin{equation}
  A =\frac{1}{h} \begin{pmatrix}
        1& -1 & 0 & \cdots & 0 \\
        -1 & 2& -1 & \cdots & 0 \\
        0 & -1& 2 & \cdots & 0 \\
        \vdots & \vdots & \vdots & \ddots & \vdots \\
        0 & 0 & \cdots& 2  & -1\\
        0 & 0 & \cdots&-1 & 1
  \end{pmatrix}\in \mathbb{R}^{(N+1)\times (N+1)}
\end{equation}

另外,由二次型的定义,计算:
\begin{equation}
  d\int_0^1 u \,dx = d\sum_{i=0}^{N}u(x_i)\int_0^1 \phi_i(x) \,dx
\end{equation}

其中:
\begin{equation}
  \int_0^1 \phi_i(x) \,dx = \frac{1}{h_i}, \quad i=1,2,\dots,N-1
\end{equation}

\begin{equation}
  \int_0^1 \phi_0(x) \,dx = \frac{1}{h_1}, \quad \int_0^1 \phi_N(x) \,dx = \frac{1}{h_N}
\end{equation}

\subsection{荷载向量}
对于荷载向量,计算:
\begin{equation}
  f_i = \int_0^1 f\phi_i \,dx \approx h f(x_i), \quad i=1,2,\dots,N-1
\end{equation}

\begin{equation}
  f_0 = \int_0^1 f\phi_0 \,dx \approx \frac{hf(x_0)}{2}, \quad 
  f_N = \int_0^1 f\phi_N \,dx \approx \frac{hf(x_N)}{2}
\end{equation}

\subsection{数值解}
那么变分问题:$ a((u,c),(v,d)) = (f,v) $ 可以转化为求解线性方程组:
\begin{equation}
  V^TKU = V^T\begin{pmatrix}
    A_{00}&\cdots&A_{0N}& \int_0^1 \phi_0 \,dx \\
    \vdots&\ddots&\vdots&\vdots \\
    A_{N0}&\cdots&A_{NN}& \int_0^1 \phi_N \,dx\\
    \int_0^1 \phi_0 \,dx&\cdots&\int_0^1 \phi_N \,dx&0
  \end{pmatrix}U = V^TF
\end{equation}
对任意 $ V=(v_0,v_1,\cdots,v_{N},d)^T $ 成立。

因此只需求解 $ KU = F $ ,即可得到 $ U= (u_0,u_1,\cdots,u_{N},c)^T $,为问题 (1) 的数值解。


\section{Results}
实验中选择 $u(x) = x^2(x-1)^2, f(x) = -12x^2+12x-2$。由于求解系数是受到下面约束限制得到的结果。
\begin{equation}
  \int_0^1 \hat{u} \,dx = 0
\end{equation}
所以在比较误差时,应当对数值解加上一个常数项,使得满足约束。即:
\begin{equation}
  \hat{u}_h = u_h + \int_0^1 u \,dx
\end{equation}
由此得到误差的定义:
\begin{equation}
  L^2 = \|\hat{u}_h - u\|_2
\end{equation}
\begin{equation}
  H^1 = \|\hat{u}_h - u\|_2 + \|\hat{u}_h' - u'\|_2
\end{equation}

结果如下:可以看出,该方法对 $ L^2 $ 误差为 2 阶,对 $ H^1 $ 误差为一阶,与前两次实验相同。
\begin{table}[htbp]
  \centering
  \begin{tabular}{|c|cc|cc|}
 \toprule
 N     & \multicolumn{1}{c}{$L^2$ error} & \multicolumn{1}{c|}{order} & \multicolumn{1}{c}{$H^1$ error} & \multicolumn{1}{c|}{order} \\
    \midrule
    10    & 3.40E-03 &       & 1.03E-01 &       \\
    20    & 8.56E-04 & 1.991 & 5.15E-02 & 0.997 \\
    40    & 2.14E-04 & 1.998 & 2.57E-02 & 1.002 \\
    80    & 5.36E-05 & 1.999 & 1.28E-02 & 1.005 \\  
    \bottomrule
    \end{tabular}%
    \label{tab:error}%
\end{table}%
\end{document}

