\documentclass[11pt]{ctexart}
% \documentclass{article}
\textheight 23.5cm \textwidth 15.8cm
%\leftskip -1cm
\topmargin -1.5cm \oddsidemargin 0.3cm \evensidemargin -0.3cm

\usepackage{verbatim}
\usepackage{fancyhdr}
\usepackage{graphicx}
\usepackage{amssymb}
\usepackage{amsmath}
\usepackage{booktabs}
\usepackage{subcaption}
\usepackage{listings}
\usepackage{color}
\usepackage{geometry}


\definecolor{mygreen}{rgb}{0,0.6,0}
\definecolor{mygray}{rgb}{0.5,0.5,0.5}
\definecolor{mymauve}{rgb}{0.58,0,0.82}

\lstset{
  language=Matlab,                % 设定语言为MATLAB
  frame=single,                   % 外围框架
  basicstyle=\footnotesize\ttfamily,   % 基本代码风格
  keywordstyle=\color{blue},      % 关键词风格
  commentstyle=\color{mygreen},   % 注释风格
  stringstyle=\color{mymauve},    % 字符串风格
  numbers=left,                   % 行号位置
  numberstyle=\tiny\color{mygray}, % 行号风格
  stepnumber=1,                   % 行号步长
  numbersep=5pt,                  % 行号与代码间隔
  backgroundcolor=\color{white},  % 代码背景色
  showspaces=false,               % 不显示空格
  showstringspaces=false,         % 不显示字符串中的空格
  showtabs=false,                 % 不显示制表符
  tabsize=4,                      % 制表符宽度
  captionpos=b,                   % 标题位置
  breaklines=true,                % 自动换行
  breakatwhitespace=false,        % 仅在空格处换行
  escapeinside={\%*}{*)},         % 可以添加LaTeX内容
}


\ctexset {
     section/format    += \sffamily\raggedright,
     subsection/format += \fbox,
}

\title{FEM Code Report2}
\author{SA24229016 王润泽}

\begin{document}
\maketitle

\section{Introduction}
编写程序求解两点边值问题:
\begin{equation}
     \begin{aligned}
          -u'' &= f, \quad 0 < x < 1, \\
          u(0) &= u(1) = 0.
     \end{aligned}
\end{equation}
选取等距网格剖分,有限元空间选取为分段连续线性多项式空间,每个单元的基函数利用单元$ I_j = [x_{j-1},x_j] $端点$x_{j-1}, x_j$得到。

\section{Method}
\subsection{基函数}
实验中采用等距网格划分,节点数为 N,节点处的值为 $ u_i ,h_i = x_{i}-x_{i-1}$ 。选取分片线性全局基函数$ \phi_i(x) $,并选取其张成的有限维线性空间进行求解,其定义如下:
\begin{equation}
     \phi_i(x) = \left\{
          \begin{aligned}
               &\frac{x-x_{i-1}}{h_i}, \quad x \in I_{i}, \\
               &\frac{x_i-x}{h_{i+1}}, \quad x \in I_{i+1}, \\
               &0, \quad \text{otherwise}.
          \end{aligned}
     \right.
\end{equation}

对于单元 $ I_i = [x_{i-1},x_i] $,其局部基函数 $ \psi $ 定义如下:
\begin{equation}
     \psi^0_i(x) = \left\{
          \begin{aligned}
               &\frac{x_i-x}{h_i}, \quad x \in I_{i}, \\
               &0, \quad \text{otherwise}.
          \end{aligned}
     \right.
\end{equation}
\begin{equation}
     \psi^1_i(x) = \left\{
          \begin{aligned}
               &\frac{x-x_{i-1}}{h_i}, \quad x \in I_{i}, \\
               &0, \quad \text{otherwise}.
          \end{aligned}
     \right.
\end{equation}

对于标准单元 $ I = [0,1] $,其基函数 $ {\varphi} $ 定义如下:
\begin{equation}
     {\varphi}^0(x) = \left\{
          \begin{aligned}
               &1-x, \quad x \in [0,1], \\
               &0, \quad \text{otherwise}.
          \end{aligned}
     \right.
\end{equation}
\begin{equation}
     {\varphi}^1(x) = \left\{
          \begin{aligned}
               &x, \quad x \in [0,1], \\
               &0, \quad \text{otherwise}.
          \end{aligned}
     \right.
\end{equation}

\subsection{刚度矩阵}
局部刚度矩阵描述了双线性形式 $ a(u,v) $ 在局部基函数上的表现,其定义如下:
\begin{equation}
     \begin{aligned}
          k_{i}^{00}&= a(\psi_i^0,\psi_i^0) = \int_{I_i} \psi_i^0(x)' \psi_i^0(x)' = \frac{1}{h_i}, \\
          k_{i}^{01}&= k_{i}^{10}= a(\psi_i^0,\psi_i^1) = \int_{I_i} \psi_i^0(x)' \psi_i^1(x)' = -\frac{1}{h_i}, \\
          k_{i}^{11}&= a(\psi_i^1,\psi_i^1) = \int_{I_i} \psi_i^1(x)' \psi_i^1(x)' = \frac{1}{h_i}.
     \end{aligned}
\end{equation}

对于等分剖分的单元,其刚度矩阵 $ K$ 定义如下:
\begin{equation}
     k_i = \frac{1}{h} \begin{pmatrix}
          1 & -1 \\
          -1 & 1
     \end{pmatrix}.
\end{equation}

对于内部单元 $ I_i = [x_{i-1},x_i] $,其全局刚度矩阵 $ K_i $ 定义如下:
\begin{equation}
     K_i = \bordermatrix{
          &       &        & {i-1}  & i      &        &       \cr
          &0      & \cdots & 0      & 0      & \cdots & 0      \cr
          &\vdots & \ddots & \vdots & \vdots & \ddots & \vdots \cr
          &0      & \cdots & 0      & 0      & \cdots & 0      \cr
          i-1&0      & \cdots & \frac{1}{h} & -\frac{1}{h} & \cdots & 0    \cr
          i&0      & \cdots & -\frac{1}{h} & \frac{1}{h} & \cdots & 0  \cr
          &0      & \cdots & 0      & 0      & \cdots & 0      \cr
          &\vdots & \ddots & \vdots & \vdots & \ddots & \vdots \cr
          &0      & \cdots & 0      & 0      & \cdots & 0      \cr
     }
\end{equation}
对于边界上 $ I_1=[0,x_1] $ 和 $ I_N=[x_{N-1},1] $ 的单元,其全局刚度矩阵 $ K_1 $ 和 $ K_N $ 定义如下:
\begin{equation}
     K_1 = \frac{1}{h} \begin{pmatrix}
          1 & 0& \cdots & 0 \\
          0 & 0& \cdots & 0 \\
          \vdots & \vdots & \ddots & \vdots \\
          0 & 0 & \cdots & 0
     \end{pmatrix}
\end{equation}
\begin{equation}
     K_{N+1} = \frac{1}{h} \begin{pmatrix}
          0 & 0& \cdots & 0 \\
          0 & 0& \cdots & 0 \\
          \vdots & \vdots & \ddots & \vdots \\
          0 & 0 & \cdots & 1
     \end{pmatrix}
\end{equation}
 
得到全局刚度矩阵 $ A $ 为:
\begin{equation}
     A = \sum_{i=1}^{N+1} K_i\
\end{equation}

\subsection{载荷向量}
取 $ f(x) = -(2\cos{x}-(x-1)\sin(x)), u(x) = (x-1)\sin{x} $,得到载荷向量 $ F $ 分量为:
\begin{equation}
     \begin{aligned}
          f_i &= (f, \phi_i) = \int_0^1 f \phi_i \, dx \\
          &= 4(hj-1)\sin^2(h/2)sin(hj)/h-2\sin(h)\cos(hj)    
     \end{aligned}
\end{equation}

\section{Results}
实验得到的结果如下:
\begin{table}[htbp]
	\centering
	\caption{误差分析}
	  \begin{tabular}{|c|cc|cc|}
	  \toprule
	  N     & \multicolumn{1}{c}{$L^2$ error} & \multicolumn{1}{c|}{order} & \multicolumn{1}{c}{$H^1$ error} & \multicolumn{1}{c|}{order} \\
	  \midrule
	  10    & $1.70\times 10^{-3}$ &   -    & $5.39\times 10^{-2}$ & - \\
	  20    & $4.26\times 10^{-4}$ & 2.00 & $2.69\times 10^{-2}$ & 1.00 \\
	  40    & $1.07\times 10^{-4}$ & 2.00 & $1.34\times 10^{-2}$ & 1.00 \\
	  80    & $2.66\times 10^{-5}$ & 2.00 & $6.68\times 10^{-3}$ & 1.01 \\
	  \bottomrule
	  \end{tabular}%
	\label{tab:error}%
  \end{table}%

\section{Discussion}
本次实验与实验1相比,主要要区别在于从不同的角度构造刚度矩阵,实验1是以全局基函数来构造刚度矩阵,而实验2是从每个单元的局部基函数出发构造刚度矩阵。从结果来看,二者的误差收敛结果是一致的,且误差随着网格的细化而减小,误差的收敛阶数也是符合理论预期的,即 $ L^2 $ 误差和 $ H^1 $ 误差的收敛阶数分别为 2 和 1。

\appendix
\section{Computer Code}
\begin{lstlisting}[caption={FEM Solver}]
     function [x, u_h] = fem_solver2(N,F_load)
     % Mesh generation
     a = 0; b = 1;  % Interval [0,1]
     h = (b - a) / N;  % Step size
     x = linspace(a, b, N+1);  % Mesh points
     A = sparse(N-1,N-1);
     % Construct the sparse matrix A
     for i = 2:N-1
         A(i-1,i-1) = A(i-1,i-1) + 1/h;
         A(i-1,i) = A(i-1,i) - 1/h;
         A(i,i-1) = A(i,i-1) - 1/h;
         A(i,i) = A(i,i) + 1/h;
     end
     A(1,1) = A(1,1) + 1/h;
     A(N-1,N-1) = A(N-1,N-1) + 1/h;
     % Construct the load vector F
     F = zeros(N-1, 1);
     for j = 1:N-1
         F(j) = F_load(j,h);
     end
     % Solve the linear system K * u_h = F
     u_h = A \ F;
     % Apply boundary conditions u(0) = 0 and u(1) = 0
     u_h = [0, u_h', 0];
 end 
\end{lstlisting}
\end{document}